\documentclass[11pt]{article}

    \usepackage[breakable]{tcolorbox}
    \usepackage{parskip} % Stop auto-indenting (to mimic markdown behaviour)
    
    \usepackage{iftex}
    \ifPDFTeX
    	\usepackage[T1]{fontenc}
    	\usepackage{mathpazo}
    \else
    	\usepackage{fontspec}
    \fi

    % Basic figure setup, for now with no caption control since it's done
    % automatically by Pandoc (which extracts ![](path) syntax from Markdown).
    \usepackage{graphicx}
    % Maintain compatibility with old templates. Remove in nbconvert 6.0
    \let\Oldincludegraphics\includegraphics
    % Ensure that by default, figures have no caption (until we provide a
    % proper Figure object with a Caption API and a way to capture that
    % in the conversion process - todo).
    \usepackage{caption}
    \DeclareCaptionFormat{nocaption}{}
    \captionsetup{format=nocaption,aboveskip=0pt,belowskip=0pt}

    \usepackage{float}
    \floatplacement{figure}{H} % forces figures to be placed at the correct location
    \usepackage{xcolor} % Allow colors to be defined
    \usepackage{enumerate} % Needed for markdown enumerations to work
    \usepackage{geometry} % Used to adjust the document margins
    \usepackage{amsmath} % Equations
    \usepackage{amssymb} % Equations
    \usepackage{textcomp} % defines textquotesingle
    % Hack from http://tex.stackexchange.com/a/47451/13684:
    \AtBeginDocument{%
        \def\PYZsq{\textquotesingle}% Upright quotes in Pygmentized code
    }
    \usepackage{upquote} % Upright quotes for verbatim code
    \usepackage{eurosym} % defines \euro
    \usepackage[mathletters]{ucs} % Extended unicode (utf-8) support
    \usepackage{fancyvrb} % verbatim replacement that allows latex
    \usepackage{grffile} % extends the file name processing of package graphics 
                         % to support a larger range
    \makeatletter % fix for old versions of grffile with XeLaTeX
    \@ifpackagelater{grffile}{2019/11/01}
    {
      % Do nothing on new versions
    }
    {
      \def\Gread@@xetex#1{%
        \IfFileExists{"\Gin@base".bb}%
        {\Gread@eps{\Gin@base.bb}}%
        {\Gread@@xetex@aux#1}%
      }
    }
    \makeatother
    \usepackage[Export]{adjustbox} % Used to constrain images to a maximum size
    \adjustboxset{max size={0.9\linewidth}{0.9\paperheight}}

    % The hyperref package gives us a pdf with properly built
    % internal navigation ('pdf bookmarks' for the table of contents,
    % internal cross-reference links, web links for URLs, etc.)
    \usepackage{hyperref}
    % The default LaTeX title has an obnoxious amount of whitespace. By default,
    % titling removes some of it. It also provides customization options.
    \usepackage{titling}
    \usepackage{longtable} % longtable support required by pandoc >1.10
    \usepackage{booktabs}  % table support for pandoc > 1.12.2
    \usepackage[inline]{enumitem} % IRkernel/repr support (it uses the enumerate* environment)
    \usepackage[normalem]{ulem} % ulem is needed to support strikethroughs (\sout)
                                % normalem makes italics be italics, not underlines
    \usepackage{mathrsfs}
    

    
    % Colors for the hyperref package
    \definecolor{urlcolor}{rgb}{0,.145,.698}
    \definecolor{linkcolor}{rgb}{.71,0.21,0.01}
    \definecolor{citecolor}{rgb}{.12,.54,.11}

    % ANSI colors
    \definecolor{ansi-black}{HTML}{3E424D}
    \definecolor{ansi-black-intense}{HTML}{282C36}
    \definecolor{ansi-red}{HTML}{E75C58}
    \definecolor{ansi-red-intense}{HTML}{B22B31}
    \definecolor{ansi-green}{HTML}{00A250}
    \definecolor{ansi-green-intense}{HTML}{007427}
    \definecolor{ansi-yellow}{HTML}{DDB62B}
    \definecolor{ansi-yellow-intense}{HTML}{B27D12}
    \definecolor{ansi-blue}{HTML}{208FFB}
    \definecolor{ansi-blue-intense}{HTML}{0065CA}
    \definecolor{ansi-magenta}{HTML}{D160C4}
    \definecolor{ansi-magenta-intense}{HTML}{A03196}
    \definecolor{ansi-cyan}{HTML}{60C6C8}
    \definecolor{ansi-cyan-intense}{HTML}{258F8F}
    \definecolor{ansi-white}{HTML}{C5C1B4}
    \definecolor{ansi-white-intense}{HTML}{A1A6B2}
    \definecolor{ansi-default-inverse-fg}{HTML}{FFFFFF}
    \definecolor{ansi-default-inverse-bg}{HTML}{000000}

    % common color for the border for error outputs.
    \definecolor{outerrorbackground}{HTML}{FFDFDF}

    % commands and environments needed by pandoc snippets
    % extracted from the output of `pandoc -s`
    \providecommand{\tightlist}{%
      \setlength{\itemsep}{0pt}\setlength{\parskip}{0pt}}
    \DefineVerbatimEnvironment{Highlighting}{Verbatim}{commandchars=\\\{\}}
    % Add ',fontsize=\small' for more characters per line
    \newenvironment{Shaded}{}{}
    \newcommand{\KeywordTok}[1]{\textcolor[rgb]{0.00,0.44,0.13}{\textbf{{#1}}}}
    \newcommand{\DataTypeTok}[1]{\textcolor[rgb]{0.56,0.13,0.00}{{#1}}}
    \newcommand{\DecValTok}[1]{\textcolor[rgb]{0.25,0.63,0.44}{{#1}}}
    \newcommand{\BaseNTok}[1]{\textcolor[rgb]{0.25,0.63,0.44}{{#1}}}
    \newcommand{\FloatTok}[1]{\textcolor[rgb]{0.25,0.63,0.44}{{#1}}}
    \newcommand{\CharTok}[1]{\textcolor[rgb]{0.25,0.44,0.63}{{#1}}}
    \newcommand{\StringTok}[1]{\textcolor[rgb]{0.25,0.44,0.63}{{#1}}}
    \newcommand{\CommentTok}[1]{\textcolor[rgb]{0.38,0.63,0.69}{\textit{{#1}}}}
    \newcommand{\OtherTok}[1]{\textcolor[rgb]{0.00,0.44,0.13}{{#1}}}
    \newcommand{\AlertTok}[1]{\textcolor[rgb]{1.00,0.00,0.00}{\textbf{{#1}}}}
    \newcommand{\FunctionTok}[1]{\textcolor[rgb]{0.02,0.16,0.49}{{#1}}}
    \newcommand{\RegionMarkerTok}[1]{{#1}}
    \newcommand{\ErrorTok}[1]{\textcolor[rgb]{1.00,0.00,0.00}{\textbf{{#1}}}}
    \newcommand{\NormalTok}[1]{{#1}}
    
    % Additional commands for more recent versions of Pandoc
    \newcommand{\ConstantTok}[1]{\textcolor[rgb]{0.53,0.00,0.00}{{#1}}}
    \newcommand{\SpecialCharTok}[1]{\textcolor[rgb]{0.25,0.44,0.63}{{#1}}}
    \newcommand{\VerbatimStringTok}[1]{\textcolor[rgb]{0.25,0.44,0.63}{{#1}}}
    \newcommand{\SpecialStringTok}[1]{\textcolor[rgb]{0.73,0.40,0.53}{{#1}}}
    \newcommand{\ImportTok}[1]{{#1}}
    \newcommand{\DocumentationTok}[1]{\textcolor[rgb]{0.73,0.13,0.13}{\textit{{#1}}}}
    \newcommand{\AnnotationTok}[1]{\textcolor[rgb]{0.38,0.63,0.69}{\textbf{\textit{{#1}}}}}
    \newcommand{\CommentVarTok}[1]{\textcolor[rgb]{0.38,0.63,0.69}{\textbf{\textit{{#1}}}}}
    \newcommand{\VariableTok}[1]{\textcolor[rgb]{0.10,0.09,0.49}{{#1}}}
    \newcommand{\ControlFlowTok}[1]{\textcolor[rgb]{0.00,0.44,0.13}{\textbf{{#1}}}}
    \newcommand{\OperatorTok}[1]{\textcolor[rgb]{0.40,0.40,0.40}{{#1}}}
    \newcommand{\BuiltInTok}[1]{{#1}}
    \newcommand{\ExtensionTok}[1]{{#1}}
    \newcommand{\PreprocessorTok}[1]{\textcolor[rgb]{0.74,0.48,0.00}{{#1}}}
    \newcommand{\AttributeTok}[1]{\textcolor[rgb]{0.49,0.56,0.16}{{#1}}}
    \newcommand{\InformationTok}[1]{\textcolor[rgb]{0.38,0.63,0.69}{\textbf{\textit{{#1}}}}}
    \newcommand{\WarningTok}[1]{\textcolor[rgb]{0.38,0.63,0.69}{\textbf{\textit{{#1}}}}}
    
    
    % Define a nice break command that doesn't care if a line doesn't already
    % exist.
    \def\br{\hspace*{\fill} \\* }
    % Math Jax compatibility definitions
    \def\gt{>}
    \def\lt{<}
    \let\Oldtex\TeX
    \let\Oldlatex\LaTeX
    \renewcommand{\TeX}{\textrm{\Oldtex}}
    \renewcommand{\LaTeX}{\textrm{\Oldlatex}}
    % Document parameters
    % Document title
    \title{HW3\_CRDM-Probability\_randomness\_and\_the\_risk\_of\_de-anonymization}
    
    
    
    
    
% Pygments definitions
\makeatletter
\def\PY@reset{\let\PY@it=\relax \let\PY@bf=\relax%
    \let\PY@ul=\relax \let\PY@tc=\relax%
    \let\PY@bc=\relax \let\PY@ff=\relax}
\def\PY@tok#1{\csname PY@tok@#1\endcsname}
\def\PY@toks#1+{\ifx\relax#1\empty\else%
    \PY@tok{#1}\expandafter\PY@toks\fi}
\def\PY@do#1{\PY@bc{\PY@tc{\PY@ul{%
    \PY@it{\PY@bf{\PY@ff{#1}}}}}}}
\def\PY#1#2{\PY@reset\PY@toks#1+\relax+\PY@do{#2}}

\expandafter\def\csname PY@tok@w\endcsname{\def\PY@tc##1{\textcolor[rgb]{0.73,0.73,0.73}{##1}}}
\expandafter\def\csname PY@tok@c\endcsname{\let\PY@it=\textit\def\PY@tc##1{\textcolor[rgb]{0.25,0.50,0.50}{##1}}}
\expandafter\def\csname PY@tok@cp\endcsname{\def\PY@tc##1{\textcolor[rgb]{0.74,0.48,0.00}{##1}}}
\expandafter\def\csname PY@tok@k\endcsname{\let\PY@bf=\textbf\def\PY@tc##1{\textcolor[rgb]{0.00,0.50,0.00}{##1}}}
\expandafter\def\csname PY@tok@kp\endcsname{\def\PY@tc##1{\textcolor[rgb]{0.00,0.50,0.00}{##1}}}
\expandafter\def\csname PY@tok@kt\endcsname{\def\PY@tc##1{\textcolor[rgb]{0.69,0.00,0.25}{##1}}}
\expandafter\def\csname PY@tok@o\endcsname{\def\PY@tc##1{\textcolor[rgb]{0.40,0.40,0.40}{##1}}}
\expandafter\def\csname PY@tok@ow\endcsname{\let\PY@bf=\textbf\def\PY@tc##1{\textcolor[rgb]{0.67,0.13,1.00}{##1}}}
\expandafter\def\csname PY@tok@nb\endcsname{\def\PY@tc##1{\textcolor[rgb]{0.00,0.50,0.00}{##1}}}
\expandafter\def\csname PY@tok@nf\endcsname{\def\PY@tc##1{\textcolor[rgb]{0.00,0.00,1.00}{##1}}}
\expandafter\def\csname PY@tok@nc\endcsname{\let\PY@bf=\textbf\def\PY@tc##1{\textcolor[rgb]{0.00,0.00,1.00}{##1}}}
\expandafter\def\csname PY@tok@nn\endcsname{\let\PY@bf=\textbf\def\PY@tc##1{\textcolor[rgb]{0.00,0.00,1.00}{##1}}}
\expandafter\def\csname PY@tok@ne\endcsname{\let\PY@bf=\textbf\def\PY@tc##1{\textcolor[rgb]{0.82,0.25,0.23}{##1}}}
\expandafter\def\csname PY@tok@nv\endcsname{\def\PY@tc##1{\textcolor[rgb]{0.10,0.09,0.49}{##1}}}
\expandafter\def\csname PY@tok@no\endcsname{\def\PY@tc##1{\textcolor[rgb]{0.53,0.00,0.00}{##1}}}
\expandafter\def\csname PY@tok@nl\endcsname{\def\PY@tc##1{\textcolor[rgb]{0.63,0.63,0.00}{##1}}}
\expandafter\def\csname PY@tok@ni\endcsname{\let\PY@bf=\textbf\def\PY@tc##1{\textcolor[rgb]{0.60,0.60,0.60}{##1}}}
\expandafter\def\csname PY@tok@na\endcsname{\def\PY@tc##1{\textcolor[rgb]{0.49,0.56,0.16}{##1}}}
\expandafter\def\csname PY@tok@nt\endcsname{\let\PY@bf=\textbf\def\PY@tc##1{\textcolor[rgb]{0.00,0.50,0.00}{##1}}}
\expandafter\def\csname PY@tok@nd\endcsname{\def\PY@tc##1{\textcolor[rgb]{0.67,0.13,1.00}{##1}}}
\expandafter\def\csname PY@tok@s\endcsname{\def\PY@tc##1{\textcolor[rgb]{0.73,0.13,0.13}{##1}}}
\expandafter\def\csname PY@tok@sd\endcsname{\let\PY@it=\textit\def\PY@tc##1{\textcolor[rgb]{0.73,0.13,0.13}{##1}}}
\expandafter\def\csname PY@tok@si\endcsname{\let\PY@bf=\textbf\def\PY@tc##1{\textcolor[rgb]{0.73,0.40,0.53}{##1}}}
\expandafter\def\csname PY@tok@se\endcsname{\let\PY@bf=\textbf\def\PY@tc##1{\textcolor[rgb]{0.73,0.40,0.13}{##1}}}
\expandafter\def\csname PY@tok@sr\endcsname{\def\PY@tc##1{\textcolor[rgb]{0.73,0.40,0.53}{##1}}}
\expandafter\def\csname PY@tok@ss\endcsname{\def\PY@tc##1{\textcolor[rgb]{0.10,0.09,0.49}{##1}}}
\expandafter\def\csname PY@tok@sx\endcsname{\def\PY@tc##1{\textcolor[rgb]{0.00,0.50,0.00}{##1}}}
\expandafter\def\csname PY@tok@m\endcsname{\def\PY@tc##1{\textcolor[rgb]{0.40,0.40,0.40}{##1}}}
\expandafter\def\csname PY@tok@gh\endcsname{\let\PY@bf=\textbf\def\PY@tc##1{\textcolor[rgb]{0.00,0.00,0.50}{##1}}}
\expandafter\def\csname PY@tok@gu\endcsname{\let\PY@bf=\textbf\def\PY@tc##1{\textcolor[rgb]{0.50,0.00,0.50}{##1}}}
\expandafter\def\csname PY@tok@gd\endcsname{\def\PY@tc##1{\textcolor[rgb]{0.63,0.00,0.00}{##1}}}
\expandafter\def\csname PY@tok@gi\endcsname{\def\PY@tc##1{\textcolor[rgb]{0.00,0.63,0.00}{##1}}}
\expandafter\def\csname PY@tok@gr\endcsname{\def\PY@tc##1{\textcolor[rgb]{1.00,0.00,0.00}{##1}}}
\expandafter\def\csname PY@tok@ge\endcsname{\let\PY@it=\textit}
\expandafter\def\csname PY@tok@gs\endcsname{\let\PY@bf=\textbf}
\expandafter\def\csname PY@tok@gp\endcsname{\let\PY@bf=\textbf\def\PY@tc##1{\textcolor[rgb]{0.00,0.00,0.50}{##1}}}
\expandafter\def\csname PY@tok@go\endcsname{\def\PY@tc##1{\textcolor[rgb]{0.53,0.53,0.53}{##1}}}
\expandafter\def\csname PY@tok@gt\endcsname{\def\PY@tc##1{\textcolor[rgb]{0.00,0.27,0.87}{##1}}}
\expandafter\def\csname PY@tok@err\endcsname{\def\PY@bc##1{\setlength{\fboxsep}{0pt}\fcolorbox[rgb]{1.00,0.00,0.00}{1,1,1}{\strut ##1}}}
\expandafter\def\csname PY@tok@kc\endcsname{\let\PY@bf=\textbf\def\PY@tc##1{\textcolor[rgb]{0.00,0.50,0.00}{##1}}}
\expandafter\def\csname PY@tok@kd\endcsname{\let\PY@bf=\textbf\def\PY@tc##1{\textcolor[rgb]{0.00,0.50,0.00}{##1}}}
\expandafter\def\csname PY@tok@kn\endcsname{\let\PY@bf=\textbf\def\PY@tc##1{\textcolor[rgb]{0.00,0.50,0.00}{##1}}}
\expandafter\def\csname PY@tok@kr\endcsname{\let\PY@bf=\textbf\def\PY@tc##1{\textcolor[rgb]{0.00,0.50,0.00}{##1}}}
\expandafter\def\csname PY@tok@bp\endcsname{\def\PY@tc##1{\textcolor[rgb]{0.00,0.50,0.00}{##1}}}
\expandafter\def\csname PY@tok@fm\endcsname{\def\PY@tc##1{\textcolor[rgb]{0.00,0.00,1.00}{##1}}}
\expandafter\def\csname PY@tok@vc\endcsname{\def\PY@tc##1{\textcolor[rgb]{0.10,0.09,0.49}{##1}}}
\expandafter\def\csname PY@tok@vg\endcsname{\def\PY@tc##1{\textcolor[rgb]{0.10,0.09,0.49}{##1}}}
\expandafter\def\csname PY@tok@vi\endcsname{\def\PY@tc##1{\textcolor[rgb]{0.10,0.09,0.49}{##1}}}
\expandafter\def\csname PY@tok@vm\endcsname{\def\PY@tc##1{\textcolor[rgb]{0.10,0.09,0.49}{##1}}}
\expandafter\def\csname PY@tok@sa\endcsname{\def\PY@tc##1{\textcolor[rgb]{0.73,0.13,0.13}{##1}}}
\expandafter\def\csname PY@tok@sb\endcsname{\def\PY@tc##1{\textcolor[rgb]{0.73,0.13,0.13}{##1}}}
\expandafter\def\csname PY@tok@sc\endcsname{\def\PY@tc##1{\textcolor[rgb]{0.73,0.13,0.13}{##1}}}
\expandafter\def\csname PY@tok@dl\endcsname{\def\PY@tc##1{\textcolor[rgb]{0.73,0.13,0.13}{##1}}}
\expandafter\def\csname PY@tok@s2\endcsname{\def\PY@tc##1{\textcolor[rgb]{0.73,0.13,0.13}{##1}}}
\expandafter\def\csname PY@tok@sh\endcsname{\def\PY@tc##1{\textcolor[rgb]{0.73,0.13,0.13}{##1}}}
\expandafter\def\csname PY@tok@s1\endcsname{\def\PY@tc##1{\textcolor[rgb]{0.73,0.13,0.13}{##1}}}
\expandafter\def\csname PY@tok@mb\endcsname{\def\PY@tc##1{\textcolor[rgb]{0.40,0.40,0.40}{##1}}}
\expandafter\def\csname PY@tok@mf\endcsname{\def\PY@tc##1{\textcolor[rgb]{0.40,0.40,0.40}{##1}}}
\expandafter\def\csname PY@tok@mh\endcsname{\def\PY@tc##1{\textcolor[rgb]{0.40,0.40,0.40}{##1}}}
\expandafter\def\csname PY@tok@mi\endcsname{\def\PY@tc##1{\textcolor[rgb]{0.40,0.40,0.40}{##1}}}
\expandafter\def\csname PY@tok@il\endcsname{\def\PY@tc##1{\textcolor[rgb]{0.40,0.40,0.40}{##1}}}
\expandafter\def\csname PY@tok@mo\endcsname{\def\PY@tc##1{\textcolor[rgb]{0.40,0.40,0.40}{##1}}}
\expandafter\def\csname PY@tok@ch\endcsname{\let\PY@it=\textit\def\PY@tc##1{\textcolor[rgb]{0.25,0.50,0.50}{##1}}}
\expandafter\def\csname PY@tok@cm\endcsname{\let\PY@it=\textit\def\PY@tc##1{\textcolor[rgb]{0.25,0.50,0.50}{##1}}}
\expandafter\def\csname PY@tok@cpf\endcsname{\let\PY@it=\textit\def\PY@tc##1{\textcolor[rgb]{0.25,0.50,0.50}{##1}}}
\expandafter\def\csname PY@tok@c1\endcsname{\let\PY@it=\textit\def\PY@tc##1{\textcolor[rgb]{0.25,0.50,0.50}{##1}}}
\expandafter\def\csname PY@tok@cs\endcsname{\let\PY@it=\textit\def\PY@tc##1{\textcolor[rgb]{0.25,0.50,0.50}{##1}}}

\def\PYZbs{\char`\\}
\def\PYZus{\char`\_}
\def\PYZob{\char`\{}
\def\PYZcb{\char`\}}
\def\PYZca{\char`\^}
\def\PYZam{\char`\&}
\def\PYZlt{\char`\<}
\def\PYZgt{\char`\>}
\def\PYZsh{\char`\#}
\def\PYZpc{\char`\%}
\def\PYZdl{\char`\$}
\def\PYZhy{\char`\-}
\def\PYZsq{\char`\'}
\def\PYZdq{\char`\"}
\def\PYZti{\char`\~}
% for compatibility with earlier versions
\def\PYZat{@}
\def\PYZlb{[}
\def\PYZrb{]}
\makeatother


    % For linebreaks inside Verbatim environment from package fancyvrb. 
    \makeatletter
        \newbox\Wrappedcontinuationbox 
        \newbox\Wrappedvisiblespacebox 
        \newcommand*\Wrappedvisiblespace {\textcolor{red}{\textvisiblespace}} 
        \newcommand*\Wrappedcontinuationsymbol {\textcolor{red}{\llap{\tiny$\m@th\hookrightarrow$}}} 
        \newcommand*\Wrappedcontinuationindent {3ex } 
        \newcommand*\Wrappedafterbreak {\kern\Wrappedcontinuationindent\copy\Wrappedcontinuationbox} 
        % Take advantage of the already applied Pygments mark-up to insert 
        % potential linebreaks for TeX processing. 
        %        {, <, #, %, $, ' and ": go to next line. 
        %        _, }, ^, &, >, - and ~: stay at end of broken line. 
        % Use of \textquotesingle for straight quote. 
        \newcommand*\Wrappedbreaksatspecials {% 
            \def\PYGZus{\discretionary{\char`\_}{\Wrappedafterbreak}{\char`\_}}% 
            \def\PYGZob{\discretionary{}{\Wrappedafterbreak\char`\{}{\char`\{}}% 
            \def\PYGZcb{\discretionary{\char`\}}{\Wrappedafterbreak}{\char`\}}}% 
            \def\PYGZca{\discretionary{\char`\^}{\Wrappedafterbreak}{\char`\^}}% 
            \def\PYGZam{\discretionary{\char`\&}{\Wrappedafterbreak}{\char`\&}}% 
            \def\PYGZlt{\discretionary{}{\Wrappedafterbreak\char`\<}{\char`\<}}% 
            \def\PYGZgt{\discretionary{\char`\>}{\Wrappedafterbreak}{\char`\>}}% 
            \def\PYGZsh{\discretionary{}{\Wrappedafterbreak\char`\#}{\char`\#}}% 
            \def\PYGZpc{\discretionary{}{\Wrappedafterbreak\char`\%}{\char`\%}}% 
            \def\PYGZdl{\discretionary{}{\Wrappedafterbreak\char`\$}{\char`\$}}% 
            \def\PYGZhy{\discretionary{\char`\-}{\Wrappedafterbreak}{\char`\-}}% 
            \def\PYGZsq{\discretionary{}{\Wrappedafterbreak\textquotesingle}{\textquotesingle}}% 
            \def\PYGZdq{\discretionary{}{\Wrappedafterbreak\char`\"}{\char`\"}}% 
            \def\PYGZti{\discretionary{\char`\~}{\Wrappedafterbreak}{\char`\~}}% 
        } 
        % Some characters . , ; ? ! / are not pygmentized. 
        % This macro makes them "active" and they will insert potential linebreaks 
        \newcommand*\Wrappedbreaksatpunct {% 
            \lccode`\~`\.\lowercase{\def~}{\discretionary{\hbox{\char`\.}}{\Wrappedafterbreak}{\hbox{\char`\.}}}% 
            \lccode`\~`\,\lowercase{\def~}{\discretionary{\hbox{\char`\,}}{\Wrappedafterbreak}{\hbox{\char`\,}}}% 
            \lccode`\~`\;\lowercase{\def~}{\discretionary{\hbox{\char`\;}}{\Wrappedafterbreak}{\hbox{\char`\;}}}% 
            \lccode`\~`\:\lowercase{\def~}{\discretionary{\hbox{\char`\:}}{\Wrappedafterbreak}{\hbox{\char`\:}}}% 
            \lccode`\~`\?\lowercase{\def~}{\discretionary{\hbox{\char`\?}}{\Wrappedafterbreak}{\hbox{\char`\?}}}% 
            \lccode`\~`\!\lowercase{\def~}{\discretionary{\hbox{\char`\!}}{\Wrappedafterbreak}{\hbox{\char`\!}}}% 
            \lccode`\~`\/\lowercase{\def~}{\discretionary{\hbox{\char`\/}}{\Wrappedafterbreak}{\hbox{\char`\/}}}% 
            \catcode`\.\active
            \catcode`\,\active 
            \catcode`\;\active
            \catcode`\:\active
            \catcode`\?\active
            \catcode`\!\active
            \catcode`\/\active 
            \lccode`\~`\~ 	
        }
    \makeatother

    \let\OriginalVerbatim=\Verbatim
    \makeatletter
    \renewcommand{\Verbatim}[1][1]{%
        %\parskip\z@skip
        \sbox\Wrappedcontinuationbox {\Wrappedcontinuationsymbol}%
        \sbox\Wrappedvisiblespacebox {\FV@SetupFont\Wrappedvisiblespace}%
        \def\FancyVerbFormatLine ##1{\hsize\linewidth
            \vtop{\raggedright\hyphenpenalty\z@\exhyphenpenalty\z@
                \doublehyphendemerits\z@\finalhyphendemerits\z@
                \strut ##1\strut}%
        }%
        % If the linebreak is at a space, the latter will be displayed as visible
        % space at end of first line, and a continuation symbol starts next line.
        % Stretch/shrink are however usually zero for typewriter font.
        \def\FV@Space {%
            \nobreak\hskip\z@ plus\fontdimen3\font minus\fontdimen4\font
            \discretionary{\copy\Wrappedvisiblespacebox}{\Wrappedafterbreak}
            {\kern\fontdimen2\font}%
        }%
        
        % Allow breaks at special characters using \PYG... macros.
        \Wrappedbreaksatspecials
        % Breaks at punctuation characters . , ; ? ! and / need catcode=\active 	
        \OriginalVerbatim[#1,codes*=\Wrappedbreaksatpunct]%
    }
    \makeatother

    % Exact colors from NB
    \definecolor{incolor}{HTML}{303F9F}
    \definecolor{outcolor}{HTML}{D84315}
    \definecolor{cellborder}{HTML}{CFCFCF}
    \definecolor{cellbackground}{HTML}{F7F7F7}
    
    % prompt
    \makeatletter
    \newcommand{\boxspacing}{\kern\kvtcb@left@rule\kern\kvtcb@boxsep}
    \makeatother
    \newcommand{\prompt}[4]{
        {\ttfamily\llap{{\color{#2}[#3]:\hspace{3pt}#4}}\vspace{-\baselineskip}}
    }
    

    
    % Prevent overflowing lines due to hard-to-break entities
    \sloppy 
    % Setup hyperref package
    \hypersetup{
      breaklinks=true,  % so long urls are correctly broken across lines
      colorlinks=true,
      urlcolor=urlcolor,
      linkcolor=linkcolor,
      citecolor=citecolor,
      }
    % Slightly bigger margins than the latex defaults
    
    \geometry{verbose,tmargin=1in,bmargin=1in,lmargin=1in,rmargin=1in}
    
    

\begin{document}
    
    \maketitle
    
    

    
    \hypertarget{homework-3-probability-randomness-and-the-risk-of-de-anonymization}{%
\section{Homework 3: Probability, randomness, and the risk of
de-anonymization}\label{homework-3-probability-randomness-and-the-risk-of-de-anonymization}}

This homework intended to conduct a peer review that focuses on the
prevalence, sensitivity and specificity aspects of the paper
\textbf{titled} \textgreater Clinical significance of blood-brain
natriuretic peptide level measurement in the detection of heart disease
in untreated outpatients: comparison of electrocardiography, chest
radiography and echocardiography \cite{4980048/HE2LER3V}.

From the title, it is clear that the researchers intend to make a
comparison of the three tests electrocardiography, chest radiography and
echocardiography for detecting heart disease in untreated patients who
attends a hospital for treatment but not staying overnight. This work
was a collaboration of nine scholars from three different institutions
in Japan, namely, Departments of Cardiology and Pneumology; Dokkyo
University School of Medicine; Keiwa Hospital and Mori Hospital. Also,
this could mark the importance of collaborating.

\hypertarget{abstract}{%
\subsection{\texorpdfstring{\textbf{Abstract}}{Abstract}}\label{abstract}}

\begin{quote}
The aim of the present study was to compare the predictive
characteristics and cost-benefit of measuring the concentration of
blood-brain natriuretic peptide (BNP), compared with electrocardiography
(ECG), chest radiography and echocardiography, as a diagnostic test for
heart disease. The study group comprised of 130 untreated patients who
had symptoms suggestive of heart disease. According to the results of
additional examinations and follow-up checks, 86 patients were diagnosed
as having heart disease, and 44 patients were judged free of heart
disease. Positive findings in each test suggestive of heart disease were
checked in accordance with criteria, and the number of positive and
negative cases for each test was calculated. The predictive
characteristics, such as specificity, sensitivity, accuracy, positive
and negative predictive values, of each test and the cost-benefit value
were calculated and analyzed statistically. The sensitivity, specificity
and accuracy of blood BNP and echocardiography were significantly
greater than those of ECG and chest radiography. Echocardiography had a
significantly lower cost-benefit value compared with measuring blood BNP
concentration. Thus, the blood BNP concentration had significantly
higher predictive characteristics than ECG and chest radiography, and a
cost benefit value significantly greater than that of echocardiography.
\end{quote}

The abstract also indicates the intention of the study. The sample size
of 130 is indicated and with suggestive heart disease cases of 86
patients, hence the sample prior of approximately 0.661. This prevalence
suggests a high rate of heart disease on these patients. Be that as may,
studies such as \cite{4980048/2GX3J5M9} had a sample prior of about
0.027, which suggest that the heart disease are rare events.

This high prevalence in this study might be because of oversampling of
rare events, which affect the accuracy and both positive and negative
predictive values \cite{undefined}.

In the paper, we expect to find measurements of predictive
characteristics such as - specificity, - sensitivity, - accuracy, -
positive and - negative predictive

for the three tests and including the test using the blood BNP
concentration level.

    \hypertarget{methods}{%
\subsection{\texorpdfstring{\textbf{Methods}}{Methods}}\label{methods}}

\begin{quote}
The study group comprised 130 untreated patients who attended the
outpatient clinic at the Department of Cardiology and Pneumology, Dokkyo
University School of Medicine or the Department of Internal Medicine of
2 community hospitals for the first time from November 1997 to April
2001, with symptoms suggestive of heart disease: (1) palpitation, (2)
dyspnea, (3) chest discomfort, (4) chest pain, (5) lower leg edema and
(6) hypertension. Patients who had one or more of these symptoms were
suspected as having heart disease.
\end{quote}

The inclusion criteria of patients were based on whether they were
untreated and that they had at least one of the six conditions. Relaxing
the conditions to at least one was good since this allows large sample
and that we do not miss other patients. The inclusion criteria selected
130 patients. The sampling method here seems to be self-selection which
reduces the amount of time to get patients who meet the criteria for the
sample needed. However, prone to - self-selection bias, - not being
representative of the population, and/or - exaggerating some finding.

For reference see
\href{http://dissertation.laerd.com/self-selection-sampling.php\#step2}{Self-selection
sampling}.

\hypertarget{test-criteria}{%
\subsubsection{\texorpdfstring{\textbf{Test
Criteria:}}{Test Criteria:}}\label{test-criteria}}

The authors indicated that all tests were performed by experienced
experts who were unaware of the patients' medical history. This idea was
usually acceptable because it minimizes measurement bias.

\begin{quote}
BNP concentration in venous blood samples, which were taken from a
peripheral vein of the patient after 30 min of bed rest during the first
visit to the hospital, were measured by immuno-radiometric assay, using
a commercially available kit (Shionoria, Shionogi Co Ltd, Tokyo, Japan).
A concentration greater than 40 pg/ml was rated positive.
\end{quote}

A test criterion at a cut-off of 40 pg/ml, which indicates a positive
heart disease was used following the study by \cite{4980048/2GX3J5M9}.
The authors of \cite{4980048/2GX3J5M9} reported the sensitivity of 85\%
and a specificity of 92\% for heart disease detection at BNP level of 40
pg/ml; however did not indicate that it could be used as a cut-off.
Hence, this cut-off value might need further justification. + The
medical review by
\href{https://www.cham.org/HealthwiseArticle.aspx?id=ux1072\#ux1079}{Healthwise
Staff} indicates that the normal values for BNP are less than 100 pg/mL.
Although this is recent (2020) and the paper in question is 2002; the
question here is whether this cut-off could have gone up by 60 pg/ml. +
\href{https://www.cham.org/HealthwiseArticle.aspx?id=ux1072\#ux1079}{Healthwise
Staff} also provided a table of \textbf{Age- and gender-specific brain
natriuretic peptide (BNP) normal reference ranges}

There might be an idea that the BNP concentration level for heart
disease status could be decided on the bases of gender and age.

\begin{quote}
During the patient's first visit to the clinic, blood BNP measurement,
ECG, chest radiography and echocardiography were performed
\end{quote}

\begin{quote}
\textbf{Electrocardiography} - tachycardia (heart rate \textgreater90
beats/min) or bradycardia (≤50beats/min), - atrial fibrillation, -
premature ventricular systole, - ST depression of 0.01 mV or greater
and/or negative T wave in 3 or more leads, - signs of acute or old
myocardial infarction (abnormal Q-wave, ST elevation and negative T wave
or reduced R wave amplitude in 3 or more leads), - findings of left
ventricular hypertrophy (SV1+RV5 or 6=3.5mV or more), - signs of right
ventricular loading (R/S ratio \textgreater1.0 and ST depression and
negative T wave in V1, excluding cases with complete right bundle-branch
block), - second-degree or more severe atrioventricular block.
\end{quote}

\begin{quote}
\textbf{Chest radiography} - cardiothoracic ratio ≥50\%, - prominence of
pulmonary vessel shadows or signs of lung congestion, and - hydrothorax
unaccompanied by signs of inflammation in the lung fields.
\end{quote}

\begin{quote}
\textbf{Echocardiography} - the end-diastolic left ventricular diameter
\textgreater55 mm, - the left ventricular ejection fraction
\textless55\%, - left atrial diameter \textgreater40 mm, - there was at
least one Doppler test finding of moderate or more severe tricuspid
valve regurgitation, mitral valve regurgitation or aortic regurgitation,
aortic valve systolic pressure gradient ≥35 mmHg or more, or mitral
valve stenosis, - there were color Doppler finding of significant
intracardiac shunt flow, - there was asynergy of the left ventricular
wall motion, - the thickness of the ventricular septum and left
ventricular posterior wall was ≥1.0 cm or greater in diastole, - there
was right ventricular overload, - the A/E flow velocity ratio of the
left ventricular inflow tract, as determined by the Doppler method, was
≥1.0, and - there was pericardial effusion, indicated by ≥5mm of
echo-free imaging in the pericardial space.
\end{quote}

For all the three tests, the authors indicated that if at least one of
the conditions satisfied the patient has a positive heart disease
status. The questions are, - do the conditions have a similar effect on
heart disease status, or are some more aggressive? - is one enough? The
paper does not give us reason to be confident about this decision.
References are not given for all condition for the three tests but some,
hence it might be useful to provide each of them with references to
clear out the doubts.

Work by \cite{4980048/YR9EL6LB} which was before this study publication
indicates that cardiothoracic ratio should be more than 50\%, not at
least 50\%. This also ties with the present studies on
\href{https://radiopaedia.org/articles/cardiothoracic-ratio\#:~:text=The\%20cardiothoracic\%20ratio\%20is\%20measured,usually\%20deemed\%20to\%20be\%20pathologic.}{Cardiothoracic
ratio}. Moreover, the study in reference indicated 55\%
\cite{4980048/V78FUCB9}. This might indicate some inconsitancies for
this measurements of condition.

Also, some measurements among the conditions differ by gender; for
example the left atrial diameter is gender specific
\cite{4980048/57GL9ZHD}.

    \hypertarget{table-1}{%
\subsubsection{\texorpdfstring{\textbf{Table
1:}}{Table 1:}}\label{table-1}}

\begin{quote}
Patient Characteristics, Symptoms and Diagnoses
\end{quote}

Table 1 shows the imbalance samples by gender with males being 87 (57
with and 26 without heart disease) and 44 (29 with and 18 without heart
disease). This imbalance might introduce some bias in the sense that the
effect of gender on heart disease status might be due to the
distribution of patients based on gender.

Why is the number of patients with hypertension in the symptoms (3)
different from the one in diagnosis (13)?

\hypertarget{statistical-analysis}{%
\subsubsection{\texorpdfstring{\textbf{Statistical
Analysis:}}{Statistical Analysis:}}\label{statistical-analysis}}

\begin{quote}
Continuous variables were presented as mean±SD. The differences in the
basic patient characteristics between the 2 groups were analysed using
unpaired t-test for continuous variables and the chi-square test for
categorical variables. The difference between specificity, sensitivity,
accuracy, and positive and negative predictive values between the 2
groups were tested by the chi-square test. The significance of
differences in the cost--benefits was analysed by the 2-sample Wilcoxon
rank sum test. Statistical analyses were carried out using computer soft
of SAS System for Windows (SAS Institute Inc, Cary, NC, USA). A p value
\textless0.05 denoted statistical significance.
\end{quote}

Several studies indicates that gender and age are important factors of
evaluating BNP for example \cite{4980048/SXJ7IXWM},
\cite{4980048/7XLWEEHW}. Also,
\href{https://www.cham.org/HealthwiseArticle.aspx?id=ux1072\#ux1079}{Healthwise
Staff} indicated that doctors evaluate BNP results based on the age and
gender of the patient. However, it is not clear how the statistical
analysis adjust for both these factors. This might also account for the
allocation bias that was realised in Table 1. Perhaps a logistic
regression model could be of good use in this regard \cite{undefined}.

    \hypertarget{results}{%
\subsection{Results}\label{results}}

\begin{quote}
\textbf{Table 2 Comparison of the Sensitivity, Specificity, Accuracy,
Positive and Negative Predictive Values, and Cost--Benefit of the Tests
for Heart Disease}
\end{quote}

From Table 2 we focus on the specificity and sensitivity of the four
tests which we use to compute the prevalence threshold. The calculations
revealed that all test required lower prevalance threshold, in which
Echocardiography is 0\%.

    \begin{tcolorbox}[breakable, size=fbox, boxrule=1pt, pad at break*=1mm,colback=cellbackground, colframe=cellborder]
\prompt{In}{incolor}{9}{\boxspacing}
\begin{Verbatim}[commandchars=\\\{\}]
\PY{k+kn}{import} \PY{n+nn}{math}
\PY{k}{def} \PY{n+nf}{PREVTHR}\PY{p}{(}\PY{n}{Test}\PY{o}{=}\PY{l+s+s1}{\PYZsq{}}\PY{l+s+s1}{Test}\PY{l+s+s1}{\PYZsq{}}\PY{p}{,}\PY{n}{sensitivity}\PY{o}{=}\PY{l+m+mf}{0.5}\PY{p}{,} \PY{n}{specificity}\PY{o}{=}\PY{l+m+mf}{0.5}\PY{p}{)}\PY{p}{:}
    \PY{k}{if} \PY{n}{sensitivity} \PY{o}{+} \PY{n}{specificity} \PY{o}{==} \PY{l+m+mi}{1}\PY{p}{:}
        \PY{n}{prevalence\PYZus{}threshold} \PY{o}{=} \PY{l+m+mf}{0.5}
    \PY{k}{else}\PY{p}{:}
        \PY{n}{prevalence\PYZus{}threshold} \PY{o}{=} \PY{p}{(}\PY{p}{(}\PY{n}{math}\PY{o}{.}\PY{n}{sqrt}\PY{p}{(}\PY{n}{sensitivity}\PY{o}{*}\PY{p}{(}\PY{o}{\PYZhy{}}\PY{n}{specificity} \PY{o}{+} \PY{l+m+mi}{1}\PY{p}{)}\PY{p}{)} \PY{o}{\PYZhy{}} \PY{l+m+mi}{1} \PY{o}{+} \PY{n}{specificity}\PY{p}{)}\PY{o}{/}
                                \PY{p}{(}\PY{n}{sensitivity} \PY{o}{+} \PY{n}{specificity} \PY{o}{\PYZhy{}} \PY{l+m+mi}{1}\PY{p}{)}\PY{p}{)}
    
    \PY{n}{prevalence\PYZus{}threshold}\PY{o}{=}\PY{n+nb}{print}\PY{p}{(}\PY{l+s+sa}{F}\PY{l+s+s2}{\PYZdq{}}\PY{l+s+si}{\PYZob{}}\PY{n}{Test}\PY{l+s+si}{\PYZcb{}}\PY{l+s+s2}{ = }\PY{l+s+si}{\PYZob{}}\PY{n}{prevalence\PYZus{}threshold}\PY{l+s+si}{:}\PY{l+s+s2}{.2f}\PY{l+s+si}{\PYZcb{}}\PY{l+s+s2}{\PYZdq{}}\PY{p}{)}
    \PY{k}{return} \PY{n}{prevalence\PYZus{}threshold}

\PY{n+nb}{print}\PY{p}{(}\PY{l+s+s2}{\PYZdq{}}\PY{l+s+s2}{Prevalence Threshold}\PY{l+s+s2}{\PYZdq{}}\PY{p}{)}
\PY{n+nb}{print}\PY{p}{(}\PY{l+s+s1}{\PYZsq{}}\PY{l+s+s1}{\PYZhy{}\PYZhy{}\PYZhy{}\PYZhy{}\PYZhy{}\PYZhy{}\PYZhy{}\PYZhy{}\PYZhy{}\PYZhy{}\PYZhy{}\PYZhy{}\PYZhy{}\PYZhy{}\PYZhy{}\PYZhy{}\PYZhy{}\PYZhy{}\PYZhy{}\PYZhy{}\PYZhy{}\PYZhy{}\PYZhy{}\PYZhy{}\PYZhy{}\PYZhy{}\PYZhy{}\PYZhy{}\PYZhy{}\PYZhy{}\PYZhy{}\PYZhy{}}\PY{l+s+s1}{\PYZsq{}}\PY{p}{)}
\PY{n}{PREVTHR}\PY{p}{(}\PY{l+s+s1}{\PYZsq{}}\PY{l+s+s1}{Brain natriuretic peptide}\PY{l+s+s1}{\PYZsq{}}\PY{p}{,}\PY{l+m+mf}{0.9}\PY{p}{,}\PY{l+m+mf}{0.98}\PY{p}{)}\PY{p}{,} 
\PY{n}{PREVTHR}\PY{p}{(}\PY{l+s+s1}{\PYZsq{}}\PY{l+s+s1}{Electrocardiography}\PY{l+s+s1}{\PYZsq{}}\PY{p}{,}\PY{l+m+mf}{0.78}\PY{p}{,}\PY{l+m+mf}{0.66}\PY{p}{)}\PY{p}{,} 
\PY{n}{PREVTHR}\PY{p}{(}\PY{l+s+s1}{\PYZsq{}}\PY{l+s+s1}{Chest radiography}\PY{l+s+s1}{\PYZsq{}}\PY{p}{,}\PY{l+m+mf}{0.66}\PY{p}{,}\PY{l+m+mf}{0.4}\PY{p}{)}\PY{p}{,} 
\PY{n}{PREVTHR}\PY{p}{(}\PY{l+s+s1}{\PYZsq{}}\PY{l+s+s1}{Echocardiography}\PY{l+s+s1}{\PYZsq{}}\PY{p}{,}\PY{l+m+mf}{0.91}\PY{p}{,}\PY{l+m+mi}{1}\PY{p}{)}
\end{Verbatim}
\end{tcolorbox}

    \begin{Verbatim}[commandchars=\\\{\}]
Prevalence Threshold
--------------------------------
Brain natriuretic peptide = 0.13
Electrocardiography = 0.40
Chest radiography = 0.49
Echocardiography = 0.00
    \end{Verbatim}

    \hypertarget{final-remarks}{%
\subsection{Final Remarks}\label{final-remarks}}

A study by \cite{4980048/5VVTXLJB} indicated that in Japan,
approximately 1 to 2 million patients have Cardiovascar Heart Failure
per year. The total population of Japan was 127.8 million, and the
number of elderly aged 65 or older was 26.6 million (this mark
approximately the population in this paper). This means a worst-case
scenario the prevalence would be about 0.0769 in the year 2006. + This
prevalence is 8.593 times less than 0.661 from the paper in review.
Perhaps it would be a good idea to use (the prevalence = 0.0769) to
adjust the predictive values and accuracy. The notes provided by
\cite{undefined} explains how to adjust for oversampling using SAS in
order to correct the predictive values. + Of course, maybe this is
because the patients in this paper are untreated; thereby making the
condition worse, hence a large number of patients with positive heart
disease. OR, the reason for the use of 40 pg/ml of BNP level as an
indication of positive heart disease OR relaxed rules conditions that
indicates the heart disease status.

\hypertarget{references}{%
\subsection{References}\label{references}}


    % Add a bibliography block to the postdoc
    
    
    
\end{document}
